\documentclass[a4paper, twoside]{article}
\usepackage[backend=biber,style=ieee,sorting=none]{biblatex}
\usepackage{geometry}
\geometry{a4paper,total={170mm,250mm},left=20mm, top=20mm,}
\usepackage{enumerate}
\usepackage[shortlabels]{enumitem}
\usepackage{optidef}
\usepackage[normalem]{ulem}
\useunder{\uline}{\ul}{}
\addbibresource{mybibliography.bib}
\usepackage{datetime}
\newdateformat{monthyeardate}{%
  \monthname[\THEMONTH], \THEYEAR}
\date{\monthyeardate\today}
%%%%%%%%%%%%%%%%%%%%%%%%%%%%%%%%%%%%%%%%%%%%%%%%%%%%%%%%%%%%%%%%%%%%%%%%%%%%%%%%%%%%%
%%%%%%%%%%%%%%%%%%%%%%%%%%%   Enter Your info here    %%%%%%%%%%%%%%%%%%%%%%%%%%%%%%%

\author{Vishal  Karmakar}
\def\snum{40220935}
\def\coursecode{SOEN 6841}
\def\coursetitle{SOFTWARE PROJECT MANAGEMENT}
\def\assignname{TOPIC: TEAMWORK}

\title{How can I organize my team
for maximum creativity, flexibility,
and success?}

%%%%%%%%%%%%%%%%%%%%%%%%%%%%%%%%%%%%%%%%%%%%%%%%%%%%%%%%%%%%%%%%%%%%%%%%%%%%%%%%%%%%%%



\begin{document}
\begin{titlepage}

\newcommand{\HRule}{\rule{\linewidth}{0.5mm}} % Defines a new command for the horizontal lines, change thickness here

%----------------------------------------------------------------------------------------
%	LOGO SECTION
%----------------------------------------------------------------------------------------
\centering
\includegraphics[width=8cm]{title/concordia.jpeg}\\[1cm] % Include a department/university logo - this will require the graphicx package
 
%----------------------------------------------------------------------------------------

\center % Center everything on the page

%----------------------------------------------------------------------------------------
%	HEADING SECTIONS
%----------------------------------------------------------------------------------------

\textsc{\LARGE \assignname}\\[1.5cm] %assignment name
\textsc{\Large \coursecode}\\[0.5cm] %Course code 
\textsc{\large \coursetitle}\\[0.5cm]  %Course title

%----------------------------------------------------------------------------------------
%	TITLE SECTION
%----------------------------------------------------------------------------------------
\makeatletter
\HRule \\[0.4cm]
{ \huge \bfseries \@title}\\[0.4cm] % Title of your document
\HRule \\[1.5cm]
 
%----------------------------------------------------------------------------------------
%	AUTHOR SECTION
%----------------------------------------------------------------------------------------

\begin{minipage}{0.4\textwidth}
\begin{flushleft} \large
\emph{Author:}\\
\@author % Your name



\end{flushleft}
\end{minipage}
~
\begin{minipage}{0.4\textwidth}
\begin{flushright} \large
\emph{Student number:} \\
\snum \\[1.2em] 
\end{flushright}
\end{minipage}\\[2cm]
\makeatother

%----------------------------------------------------------------------------------------
%	DATE SECTION
%----------------------------------------------------------------------------------------

{\large \today}\\[2cm] % Date, change the \today to a set date if you want to be precise

\vfill % Fill the rest of the page with whitespace

\end{titlepage}

\section{Abstract}
This report explores strategies for organizing teams to maximize creativity, flexibility, and overall success, with a focus on team experience and size. Acknowledging the impact of team dynamics, particularly for new teams, the report delves into the critical stages of 'forming and storming,' emphasizing rapid progression towards 'norming' through collaborative activities such as brainstorming and team-building.

For teams with novice contributors, the report advocates for proactive mentoring and guidance, aiming to swiftly integrate less-experienced members into productive roles. Training initiatives addressing the team's collective needs are also highlighted, underlining the importance of skill-building to enhance project outcomes.

Recognizing the invaluable contributions of experienced team members, the report recommends engaging them in mentoring and training roles, leveraging their expertise. It suggests fostering a culture of appreciation through rewards for creativity and effective problem-solving. Additionally, insights from seasoned team members are instrumental in focusing discussions on project objectives and trusting their autonomy in executing tasks.



\begin{figure}[h]
\centering
\begin{minipage}{.5\textwidth}
  \centering
  \includegraphics[width=0.25\textwidth]{title/mentoring.png}
  \caption{Mentoring Program}
  \label{fig:test1}
\end{minipage}%
\begin{minipage}{.5\textwidth}
  \centering
  \includegraphics[width=0.25\textwidth]{title/ringleman.png}
  \caption{Ringelmann Effect}
  \label{fig:test2}
\end{minipage}
\end{figure}

Considering team size, the report emphasizes the effectiveness of team-building activities and creativity rewards for both small and large teams. However, as teams grow in size, it acknowledges the necessity of delegating innovation encouragement and relationship maintenance to project leaders. Program-level structures and incentives are discussed as supportive tools, but the emphasis is on selecting personable and competent project leaders and minimizing inter-project dependencies to enable independent and self-managed project teams.

This report amalgamates practical insights with theoretical foundations, drawing from experiences at Hewlett-Packard and referencing works by notable authors in organizational psychology and management. It serves as a comprehensive guide for team leaders and project managers aiming to optimize team dynamics for enhanced creativity, flexibility, and overall success.

\section{Potential Issues related to organizing a team}
\subsection{Balancing Independence and Collaboration}
\begin{itemize}
  \item \textbf{Issue}: There's a challenge in finding the right balance between promoting independence within project teams and ensuring effective collaboration.
  \item \textbf{Annotation}: Excessive independence can lead to siloed efforts, hindering cross-team collaboration and knowledge sharing.\cite{Tuckman1965}
\end{itemize}

\subsection{Recognition Disparities Among Team Members}
\begin{itemize}
  \item \textbf{Issue}: Recognizing the contributions of experienced team members might create a hierarchy, potentially undervaluing the efforts of novices.
  \item \textbf{Annotation}: Inequality in recognition can lead to demotivation and hinder a positive team culture.\cite{Deci2001}
\end{itemize}

\subsection{Scalability of Team-Building Strategies}
\begin{itemize}
  \item \textbf{Issue}: Skepticism regarding the scalability of team-building strategies, with concerns that what works for small teams may not be effective for larger ones.
  \item \textbf{Annotation}: Adapting strategies for larger teams while maintaining core principles is essential for sustaining a creative and flexible work environment.\cite{Hackman2002}
\end{itemize}

\subsection{Time and Resource Allocation for Team Development}
\begin{itemize}
  \item \textbf{Issue}: Opponents may argue that dedicating time and resources to team-building activities is a luxury, diverting attention from immediate project goals.
  \item \textbf{Annotation}: Investing in team development pays off in improved efficiency, innovation, and long-term project success.\cite{Katzenbach1993}
\end{itemize}

\subsection{Maintaining Innovation Amidst Project Constraints}
\begin{itemize}
  \item \textbf{Issue}:  Innovation might be compromised when teams face tight project schedules and resource constraints.
  \item \textbf{Annotation}: Striking a balance between meeting project deadlines and fostering a creative environment is crucial for sustained success.\cite{Amabile2011}
\end{itemize}

\subsection{Communication Challenges in Diverse Teams}
\begin{itemize}
  \item \textbf{Issue}: Diverse teams may face communication challenges due to cultural differences, potentially hindering the flow of creative ideas.
  \item \textbf{Annotation}: Bridging communication gaps is crucial for leveraging diverse perspectives and fostering innovation.\cite{Cox1994}
\end{itemize}

\subsection{Resistance to Change}
\begin{itemize}
  \item \textbf{Issue}:  Team members might resist changes in established workflows, hindering adaptability.
  \item \textbf{Annotation}: vercoming resistance requires effective change management strategies to create a culture that embraces flexibility and innovation.\cite{Kotter1996}
\end{itemize}

\subsection{Burnout and Creativity Drain}
\begin{itemize}
  \item \textbf{Issue}:  Overworking teams may lead to burnout, negatively impacting creativity and flexibility.
  \item \textbf{Annotation}: Balancing workload and promoting a healthy work-life environment is essential for sustaining creativity.\cite{Amabile2011}
\end{itemize}

\subsection{Lack of Resources for Skill Development}
\begin{itemize}
  \item \textbf{Issue}:  Insufficient resources allocated to skill development may limit the team's ability to adapt to evolving project requirements.
  \item \textbf{Annotation}: Investing in continuous training and development is crucial for building a flexible and competent team.\cite{Noe2010}
\end{itemize}

\subsection{Inadequate Recognition Systems}
\begin{itemize}
  \item \textbf{Issue}:  Inadequate systems for recognizing and rewarding creative contributions may demotivate team members.
  \item \textbf{Annotation}: Establishing a robust recognition system is vital for sustaining a culture of innovation.\cite{Scott1994}
\end{itemize}







\newpage
\printbibliography
\end{document}




